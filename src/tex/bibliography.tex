\bibliographystyle{abbrv}
\begin{thebibliography}{9}
    \bibitem{random_digits}
    RAND:
    \textit{A Million Random Digits with 100,000 Normal Deviates} (1955)
    \\\texttt{\url{https://www.rand.org/content/dam/rand/pubs/monograph_reports/MR1418/MR1418.deviates.pdf}}
    (abgerufen 2019-06-01)

    \bibitem{ferranti_rng}
    A. M. Turing, R. A. Brooker:
    \textit{Programmer's Handbook for the Manchester Electronic Computer Mark II (2. Auflage)} 1. (1952)
    \\\texttt{\url{http://curation.cs.manchester.ac.uk/computer50/www.computer50.org/kgill/mark1/progman.html}}
    (abgerufen 2019-06-01)

    \bibitem{middle_square}
    Die 1949er Papiere wurden erst 1951 neu gedruckt:
    \\J. Neumann, Von:
    \textit{Various Techniques Used in Connection with Random Digits}
    (1951, National Bureau of Standards, Applied Math Series)
    \\\texttt{\url{https://mcnp.lanl.gov/pdf_files/nbs_vonneumann.pdf}}
    (abgerufen 2019-06-01)

    \bibitem{latexcompanion}
    W. Killmann, W. Schindler; Bundesamt f\"ur Sicherheit in der Informationstechnik:
    \textit{A proposal for Functionality classes for random number generators} (2011-09-18)
    \\\texttt{\url{https://www.bsi.bund.de/SharedDocs/Downloads/DE/BSI/Zertifizierung/Interpretationen/AIS_31_Functionality_classes_for_random_number_generators_e.pdf?__blob=publicationFile}}
    (abgerufen 2019-04-30)
    %Addison-Wesley, Reading, Massachusetts, 1993.

    \bibitem{seminumerical algorithms}
    D. E. Knuth:
    \textit{The Art of Computer Programming, Volume 2: Seminumerical Algorithms} (1997, Addison-Wesley)
    \\\texttt{\url{https://doc.lagout.org/science/0_Computer%20Science/2_Algorithms/The%20Art%20of%20Computer%20Programming%20%28vol.%202_%20Seminumerical%20Algorithms%29%20%283rd%20ed.%29%20%5BKnuth%201997-11-14%5D.pdf
    }}
    (abgerufen 2019-06-01)

    \bibitem{pokemon}
    ''Fractal Fusion'':
    \textit{Pokemon RBGY Random Number Generator}
    \\\texttt{\url{https://web.archive.org/web/20190105083840/http://tasvideos.org/GameResources/GBx/Pokemon.html}}
    (abgerufen 2019-06-01)

    \bibitem{fortran}
    Compaq:
    \textit{Compaq Fortran Language Reference Manual}
    \\\texttt{\url{http://h30266.www3.hpe.com/odl/unix/progtool/cf95au56/lrm0315.htm#randu_intrin}}
    (abgerufen 2019-05-04)

    \bibitem{neumann}
    J. Neumann, Von:
    \textit{Various techniques used in connection with random digits}, 12. 36-38
    (1951, National Bureau of Standards, Applied Mathematics Series)
    \\\texttt{\url{https://dornsifecms.usc.edu/assets/sites/520/docs/VonNeumann-ams12p36-38.pdf}}
    (abgerufen 2019-04-30)

    \bibitem{einstein}
    Frank Yellin et al. f\"ur Oracle:
    \textit{Random.java} (2014-03-04, OpenJDK)
    \\\texttt{\url{http://hg.openjdk.java.net/jdk8/jdk8/jdk/file/tip/src/share/classes/java/util/Random.java}}
    (abgerufen 2019-04-30)

    \bibitem{aes}
    N. Ferguson, B. Schneier, T. Kohno:
    \textit{Cryptography Engineering: Design Principles and Practical Applications} (2010, Wiley Publishing)
    \\\texttt{\url{https://www.schneier.com/academic/paperfiles/fortuna.pdf}}
    (abgerufen 2019-06-01)

\end{thebibliography}
