Die Implementierung von effizienter \texttt{MVC}-Kodierung basiert gr\"o{\ss}tenteils auf
\textit{Efficient Compression of Multi-View Video Exploiting Inter-View Dependencies Based on \texttt{H.264/MPEG4-AVC}}
von P. Merkle, K. M\"uller, A. Smolic, und T. Wiegand\cite{paper}.
Dabei handelt es sich um eines von mehreren Papieren, welches noch vor der Standardisierung 2009\cite{mvc} von
\texttt{MVC} in \texttt{H.264} als Vorschlag f\"ur eine solche Kodierung ver\"offentlich wurde.

\noindent\\ Der H.264-Standard selbst\cite{h264}, welcher in der aktuellen Ausgabe kostenpflichtig verf\"ugbar ist, beschreibt
lediglich die allgemeine Struktur des \texttt{H.264}-Datenformats.
Dazu z\"ahlen die Arten und Abfolgen von Rahmen, die Gr\"o{\ss}e von Macrobl\"ocken, Header-Information am Anfang eines
Rahmens und vieles mehr.
Die konkrete Abfolge von Rahmen, wie z.B. die oft verwendete \texttt{IBBPBBIBBPBBI}-Abfolge\cite{frame-order} steht
hingegen dem Encoder frei zur Wahl.

\noindent\\ Am 29.10.2019 wurde \texttt{FRIM 1.00}\cite{frim} ver\"offentlicht.
Hierbei handelt es sich um einen Encoder und Decoder f\"ur \texttt{H.264} mit \texttt{MVC}, welcher f\"ur
Microsoft Windows (x86/x86\_64) verf\"ugbar ist.
Zum Experimentieren mit 3D-Videos ist dieser ein sehr geeignetes und vor allem kostenloses Werkzeug.

\noindent\\ \texttt{MVC} wird bedauernswerterweise nicht vom popul\"aren Kommandozeilenprogramm \texttt{ffmpeg}\cite{
ffmpeg}
unterst\"utzt.
So werden zus\"atzliche Ansichten beim Dekodieren schlichtweg ignoriert.

