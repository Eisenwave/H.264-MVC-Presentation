\noindent\newline Das Kodieren von meherer Perspektiven (\textit{Multi-View-Coding}) und deren separates Darstellen ist
eine popul\"are Methode um den visuellen Effekt eines dreidimensionalen Videos zu erzeugen.
Jedoch bringt dieses auch einzigartige Herausforderungen mit sich, da unsere Videosequenz um eine Raumachse erweitert
wird.

\noindent\newline Unter Nutzung von hierarchischen B-Rahmen-Baumstrukturen k\"onnen wir erfolgreich Multi-View-Coding
f\"ur \texttt{H.264} implementieren.
Unsere Implementierung ist sowohl f\"ur mehrere Perspektiven skalierbar als auch effizient im Speicher
anordenbar.
Zwar existieren Ausnahmef\"alle, jedoch sehen wir konsistente Reduktion in der Datenmenge im Vergleich
zu Simulcast.

\noindent\newline Die Flexibilit\"at des \texttt{H.264}-Codecs erlaubt es uns solche Baumstrukturen standardkonform
zu kodieren.
Bedauernswerterweise existiert au{\ss}er \texttt{FRIM}~\cite{frim} keine Implementierung von \texttt{MVC}, insbesondere
\texttt{ffmpeg}~\cite{ffmpeg} unterst\"utzt dies nicht.





